
%%%%%%%%%%%%%%%%%%%%%%%%%%%%%%%%%%%%%%%%%%%%%%%%%%%%%%%%%%%%%%%%
% Mohammad Al Ebedan                                           %
% ECE 351-52                                                   %
% Lab 3                                                        %
% Sep 22, 2021                                                  %
%%%%%%%%%%%%%%%%%%%%%%%%%%%%%%%%%%%%%%%%%%%%%%%%%%%%%%%%%%%%%%%%

\documentclass[11pt,a4]{article}
\usepackage[utf8]{inputenc}
\usepackage{fullpage}
\usepackage{hyperref}
\usepackage{listings}
\usepackage{xcolor}
\usepackage{graphicx}

\definecolor{codegreen}{rgb}{0,0.6,0}
\definecolor{codegray}{rgb}{0.5,0.5,0.5}
\definecolor{codeblue}{rgb}{0,0,0.95}
\definecolor{backcolour}{rgb}{0.95,0.95,0.92}

\lstdefinestyle{mystyle}{
	backgroundcolor=\color{backcolour},
	commentstyle=\color{codegreen},
	keywordstyle=\color{codeblue},
	numberstyle=\tiny\color{codegray},
	stringstyle=\color{codegreen},
	basicstyle=\ttfamily\footnotesize,
	breakatwhitespace=false,
	breaklines=true,
	captionpos=b,
	keepspaces=true,
	numbers=left,
	numbersep=5pt,
	showspaces=false,
	showstringspaces=false,
	showtabs=false,
	tabsize=2
}
\lstset{style=mystyle}

\title{Lab3: User-Defined Functions}								

\author{Mohammad Al Ebedan}						

\date{22/9/2021}



%%%%%%%%%%%%%%%%%%%%%%%%%%%%%%%%%%%%%%%%%%%%
\begin{document}

%%%%%%%%%%%%%%%%%%%%%%% Introduction %%%%%%%%%%%%%%%%
	\section{Introduction}
	
	The goal of this lab is to become familiar with convolution and its properties using Python. 

%%%%%%%%%%%%%%%%%%%%%%%% Equations %%%%%%%%%%%%%%%%%%%		
	\section{Equations}
	
	For this lab, we use three function from the previous lab to convolve them together in this lab;
	\begin{equation}
		funct1(t) = u(t - 2) - u(t - 9)\\
	\end{equation}
	
	\begin{equation}
		funct2(t) = e^-t\\
	\end{equation}
	
	\begin{equation}
		funct3(t) = r(t) - r(t-3) + 5u(t-3) - 2u(t-6) - 2r(t-6)\\
	\end{equation}
	
	
	
	
%%%%%%%%%%%%%%%%%%%%%%%% Methodology & Results %%%%%%%%%%%%%%%%%
	\section{Methodology and Results}
	
	Lab 3 has 2 parts:
	\begin{itemize}
		\item
%%%%%%%%%%%%%%%%%%%%%%%%%%%%%%%%%%%%%%	Part 1:
		
		In this part, we created three functions (user defined) and plotted them in a single figure (separate subplots) from 0 to 20 with time steps small enough to achieve appropriate resolution.
		\begin{lstlisting}[language=Python]
			#%% task 1
			
			#%% Part 1 Task 1
			
			def u(t):
			y = np.zeros(t.shape)
			
			for i in range(len(t)):
			if t[i] >= 0:
			y[i] = 1
			else:
			y[i] = 0
			return y
			
			def r(t):
			y = np.zeros(t.shape)
			
			for i in range(len(t)):
			if t[i] >= 0:
			y[i] = t[i]
			else:
			y[i] = 0
			return y
			
			def f_1(t):
			y = u(t - 2) - u(t - 9)
			return y
			
			def f_2(t):
			y = np.exp(-t)*u(t)
			return y
			
			def f_3(t):
			y = r(t - 2)*(u(t - 2) - u(t -3)) + r(4 - t)*(u(t - 3) - u(t - 4))
			return y
			
			#%% Part 1 Task 2
			
			steps = 1e-2
			t = np.arange(0, 20 + steps, steps)
			
			plt.figure(figsize = (10, 7))
			plt.subplot(3, 1, 1)
			plt.plot(t, f_1(t))
			plt.ylabel('f_1(t)')
			plt.title('Three User-Defined Functions')
			plt.ylim([0, 1.2])
			plt.grid()
			plt.subplot(3, 1, 2)
			plt.plot(t, f_2(t))
			plt.ylabel('f_2(t)')
			plt.ylim([0, 1.2])
			plt.grid()
			plt.subplot(3, 1, 3)
			plt.plot(t, f_3(t))
			plt.ylabel('f_3(t)')
			plt.ylim([0, 1.2])
			plt.grid()
			plt.xlabel('t [s]')
			plt.show()
			
			
		\end{lstlisting}
		
		The output of this part is:\\
		\includegraphics[width=150mm]{p1t2}\\   
		
		\item
%%%%%%%%%%%%%%%%%%%%%%%%%%%%%%%%%%%%%%	Part 2:\\
		We write our own code to perform convolution. Then used the scipy.signal.convolve() function to verify our code and compared them through plot:\\
		\begin{lstlisting}[language=Python]
			#%% Part 2 Task 1
			
			def conv(f1, f2):
			Nf1 = len(f1)
			Nf2 = len(f2)
			f1Extended = np.append(f1, np.zeros((1, Nf2 -1)))
			f2Extended = np.append(f2, np.zeros((1, Nf1 -1)))
			result = np.zeros(f1Extended.shape)
			
			for i in range(Nf2 + Nf1 - 2):
			result[i] = 0
			for j in range(Nf1):
			if(i - j + 1 > 0):
			try:
			result[i] = result[i] + f1Extended[j]*f2Extended[i - j + 1]
			except:
			print(i, j)
			return result
			
			steps = 1e-2
			t = np.arange(0, 20 + steps, steps)
			tExtended = np.arange(0, 2*t[len(t) - 1], steps)
			
			f1 = f_1(t)
			f2 = f_2(t)
			f3 = f_3(t)
			
			#%% Part 2 Task 2
			
			conv12 = conv(f1, f2)*steps
			conv12Check = sig.convolve(f1, f2)*steps
			
			plt.figure(figsize = (10, 7))
			plt.plot(tExtended, conv12, label = 'User-Defined Convolution')
			plt.plot(tExtended, conv12Check, '--', label = 'Built-In Convolution')
			plt.ylim([0, 1.2])
			plt.grid()
			plt.legend()
			plt.xlabel('t [s]')
			plt.ylabel('f_1(t) * f_2(t)')
			plt.title('Convolution of f_1 and f_2')
			plt.show()
			
			#%% Part 2 Task 3
			
			conv23 = conv(f2, f3)*steps
			conv23Check = sig.convolve(f2, f3)*steps
			
			plt.figure(figsize = (10, 7))
			plt.plot(tExtended, conv23, label = 'User-Defined Convolution')
			plt.plot(tExtended, conv23Check, '--', label = 'Built-In Convolution')
			plt.ylim([0, 1.2])
			plt.grid()
			plt.legend()
			plt.xlabel('t [s]')
			plt.ylabel('f_2(t) * f_3(t)')
			plt.title('Convolution of f_2 and f_3')
			plt.show()
			
			#%% Part 2 Task 4
			
			conv13 = conv(f1, f3)*steps
			conv13Check = sig.convolve(f1, f3)*steps
			
			plt.figure(figsize = (10, 7))
			plt.plot(tExtended, conv13, label = 'User-Defined Convolution')
			plt.plot(tExtended, conv13Check, '--', label = 'Built-In Convolution')
			plt.ylim([0, 1.2])
			plt.grid()
			plt.legend()
			plt.xlabel('t [s]')
			plt.ylabel('f_1(t) * f_3(t)')
			plt.title('Convolution of f_1 and f_3')
			plt.show()
			
		\end{lstlisting}
		
	\end{itemize}
	
	The output of this part is the following three graphs:\\

	\includegraphics[width=150mm]{p2t2}\\   
	\includegraphics[width=150mm]{p2t3}\\
	\includegraphics[width=150mm]{p2t4}\\
	\
	
%%%%%%%%%%%%%%%%%%%%%%%% Questions %%%%%%%%%%%%%%%%%%%%
	\section{Questions}
	
	\begin{enumerate}
		\item 
		Did you work alone or with classmates on this lab? If you collaborated to get to the solution,what did that process look like?\\
		\\I worked alone.\\
		
		\item
		What was the most difficult part of this lab for you, and what did your problem-solving process look like?\\
		\\To understand the user-defined code to perform convolution.\\
		
		\item
		Did you approach writing the code with analytical or graphical convolution in mind? Why did you chose this approach?\\
		\\Yes, we chose this approach because we needed to perform convolution without using integral functions.\\
		
		\item
		Leave any feedback on the clarity of lab tasks, expectations, and deliverables.\\

	\end{enumerate}

%%%%%%%%%%%%%%%%%%%%%%%% Conclusion %%%%%%%%%%%%%%%%%%%%	
	\section{Conclusion}
	
	At the end of this lab, we learned how to convolve two functions together using Python. 
	
\end{document}
